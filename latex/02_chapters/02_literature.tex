
The literature on Cross-nested logit models (CNL) is so far highly theoretical, compared to the way econometrics often highlight the practical applications of new techniques. This is in part due to the complexity of the models, and in part because many questions on for example general identifiability are still to be addressed for the CNL.
So far the CNL has found its uses primarily in traffic research and travel mode theory, where it helps researchers study individuals' choices without the restrictive limitations of the simpler choice models. Compared to econometrics these fields appear to have a stronger tradition for modelling complex systems, instead of relying on natural experiments.
\\ \\
A large part of the existing literature on the CNL model is due to Michel Bierlaire \citep{bierlaire_general_2001, bierlaire_biogeme:_2003, bierlaire_theoretical_2006, bierlaire_estimation_2008, bierlaire_estimation_2009} who has contributed both by unifying the various formulations of the CNL that have been proposed \citep{bierlaire_theoretical_2006} and by developing the software package \textit{Biogeme} (\cite{bierlaire_biogeme:_2003}) for estimation of discrete choice models. Other work on the CNL model specifically include \citet{papola_developments_2004} but in general the literature due to others than Bierlaire has been superseded by his attempts to unify the work of many others.

Bierlaire builds on the basis of the extensive work in which Daniel McFadden has introduced the \textit{Generalized Extreme Value} (GEV) model class \citep{mcfadden_modelling_1977, mcfadden_quantitative_1977, hausman_specification_1984}.
\\ \\
There is some literature on the estimation of complex network models, e.g. by \citet{newman_computational_2018} and \citet{mai_dynamic_2017} however we do not touch specifically on this in our paper as the techniques involved in efficient estimation of these models are in general to advanced to mention here.

\subsection{Software}
In terms of software there are essentially two options that are available. A couple of other alternatives are out there (even STATA can estimate nested models) but these are generally expensive and/or unable to estimate the generalized versions such as the CNL model. The oldest of the free software packages is Biogeme \citep{bierlaire_biogeme:_2003} which has now been extended to include the more user friendly PythonBiogeme for Python 3. \citep{bierlaire_pythonbiogeme:_2016}. A newer alternative, also made for Python 3.6 is Larch, presented in \citet{newman_computational_2018}. While Biogeme relies on separate model files for specification, Larch handles everything from within a single Python session, making it preferable for quick estimation. Furthermore Larch is on PyPi (Python Package Index) and has a significant speed advantage over Biogeme.
\\ \\
% The two options differ slighly in the ways data should be formatted, and in their optimization routines, but should in almost all cases be interchangable.
