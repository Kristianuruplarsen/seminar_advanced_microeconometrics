
In this paper we have presented the Cross-nested Logit (CNL), through a thorough review of less complex models in the GEV family, and a GEV-oriented description of the features and limitations of the cross nested logit model. In this regard we contribute primarily known material. We also show a number of derivations, e.g. the marginal effects in equation \eqref{eq: marginaleffects} and the properties for substitution patterns in section \ref{sec: iiaproof}, which are perhaps to concrete to have been of interest in previous work, but has a very direct link to the ordinary logit models.
\\ \\
We simulate data from both nested and cross-nested structures, and implement two relatively naive estimation techniques as computational difficulties hinder a full likelihood estimation.
\\ \\
The paper also contains discussions on the usability of the cross nested models, and implements a simple model on real data from the Danish DREAM database. In this regard we find that while there is with no doubt lessons to be learnt from complex structural modelling. However the estimates suffer in interpretability and transparency, why we foresee some waiting before the CNL gains widespread use. We find evidence that nesting is a better model specification for the data (see section \ref{subsec: dreamdata}), than MNL suggesting that those on sick leave at time $n$ will one year later act in accordance with a nested structure with a node for employment, and a nest covering various states of unemployment.
